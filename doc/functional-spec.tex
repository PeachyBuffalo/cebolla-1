\documentclass{book}

\usepackage{hyperref}

\begin{document}

\title{cebolla Functional Specification}
\author{
    Alec Ten Harmsel\\
    $<$alec@alectenharmsel.com$>$
}
\date{}
\maketitle

\chapter{Introduction}

\section{Overview}

cebolla helps individual and small shops manage projects.
Using cebolla, you can add customers, add projects, make estimates and change
orders, keep track of tasks and hours, and make invoices.

\section{Non-Goals}

cebolla makes invoices, but it doesn't do double-entry bookkeeping.
Use something like GnuCash for that instead.

\chapter{Features}

\section{Entities}

Work in progress.

\section{Customers}

% TODO
Work in progress.

\section{Projects}

Projects group related tasks and store billing-related configuration.
Projects may be estimated or ad hoc.
Estimated projects may be billed monthly or lump sum when delivered.
Estimated projects have a due date, but it's set by the estimate or latest
change order.
Ad hoc projects may only be billed monthly, and do not have a due date.
Once created, a project's name may be changed, but other attributes cannot.
A project's due date may be changed via a change order.

Projects without any estimates, tasks, or invoices may be deleted (e.g. if it
was created in error).

% TODO: projects can't be archived if any tasks are not done

\subsection{Project Lifecycle}

Ad hoc projects are created ``in progress'' and remain that way until archived.
Estimated projects start in the ``estimating'' state, transition to ``in
progress'' after the estimate is finalized, and then become ``done'' once all
the tasks are finished.
Change orders may be added while an estimated project is running and don't
change the state.

When an estimated project is created, an estimate is automatically created as
well.
Estimates have a number, a due date, and a status.
The number for the initial estimate is zero, the first change order is one, and
so on.
The due date must be set before finalizing, and will be copied to the project
when it's finalized.
The status is ``Draft'' initially, and becomes ``Active'' once finalized.

Tasks may be created for an estimated project when there's a draft estimate,
and all these tasks must have an estimated number of hours.
The estimate may be finalized by clicking the ``Finalize'' button; this will
link the tasks to the estimate, update the project's due date, and make the
estimate active.
On a draft estimate's page, the future-associated tasks are shown along with
some estimated due dates--one based on a 40 hour work week with every task
completed in its estimated time and one based on a 40 hour work week with every
task taking 20\% longer (called the buffer estimate).

The proper name for estimates with a number of zero is ``Estimate M-N'', where
``M'' is the project number and ``N'' is the estimate number.
The proper name for estimates with a number of one or greater is ``Change Order
M-N'', where ``M'' is the project number and ``N'' is the estimate number.

Estimates may be exported to PDF.
These PDFs are meant to be given to customers, and show the entity name,
project name, the project description, the estimate's proper name, and the task
subjects.
For draft estimates, the estimated 20\% buffer due date is shown as ``Estimated
Completion Date.''
For active estimates, the configured due date is shown as ``Targeted Completion
Date.''

\subsection{Tasks}

% TODO: tasks can't be added if there is no in-progress estimate or change
% order

\end{document}
